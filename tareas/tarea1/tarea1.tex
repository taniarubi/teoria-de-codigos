\documentclass[letterpaper,11pt]{article}

% Soporte para los acentos.
\usepackage[utf8]{inputenc}
\usepackage[T1]{fontenc}    
% Idioma español.
\usepackage[spanish,mexico, es-tabla]{babel}
% Soporte de símbolos adicionales (matemáticas)
\usepackage{multirow}
\usepackage{amsmath}		
\usepackage{amssymb}		
\usepackage{amsthm}
\usepackage{amsfonts}
\usepackage{latexsym}
\usepackage{enumerate}
\usepackage{ragged2e}
\usepackage{multicol}
\setcounter{MaxMatrixCols}{20}
% Modificamos los márgenes del documento.
\usepackage[lmargin=2cm,rmargin=2cm,top=2cm,bottom=2cm]{geometry}

\title{Facultad de Ciencias, UNAM \\ Teoría de Códigos \\ Tarea 1}
\author{Rubí Rojas Tania Michelle}
\date{\today}

\begin{document}
\maketitle

\begin{enumerate}
    % Ejercicio 1.
    \item Construye el campo $\mathbb{F}_{16}$. También da sus tablas de suma y 
    multiplicación.
    
    \textsc{Solución:} En el anillo $\mathbb{Z}_{2}$ existe el polinomio 
    irreducible $f(x) = x^{4}+x+1$. Tenemos que 
    \begin{equation}
        \mathbb{F}_{16} = \mathbb{Z}_{2}[x]/(x^{4}+x+1)
    \end{equation}
    
    donde los elementos de $\mathbb{F}_{16}$ son:
    \begin{align*}
        \mathbb{F}_{16} 
        &= \{ax^{3} + bx^{2} + cx + d \; : \; a,b,c,d \in \mathbb{Z}_{2}\} \\
        &= \{0, 1, x, x + 1, x^2, x^2 + 1, x^2 + x, x^2 + x + 1, x^3, 
             x^3 + 1, x^3 + x, x^3 + x + 1, x^3 + x^2, \\ 
        &\; \; \; \; \; \; \; x^3 + x^2 + x, x^3 + x^2 + 1, x^3 + x^2 + x + 1\}
    \end{align*}
    
    Etiquetamos cada uno de los elementos de $\mathbb{F}_{16}$ de la siguiente 
    manera:
    \begin{multicols}{3}
        \begin{itemize}
            \item $g_{0}(x) = 0$
            \item $g_{1}(x) = 1$
            \item $g_{2}(x) = x$
            \item $g_{3}(x) = x+1$
            \item $g_{4}(x) = x^{2}$
            \item $g_{5}(x) = x^{2}+1$
            \item $g_{6}(x) = x^{2}+x$
            \item $g_{7}(x) = x^{2}+x+1$
            \item $g_{8}(x) = x^{3}$
            \item $g_{9}(x) = x^{3}+1$
            \item $g_{10}(x) = x^{3}+x$
            \item $g_{11}(x) = x^{3}+x+1$
            \item $g_{12}(x) = x^{3}+x^{2}$
            \item $g_{13}(x) = x^{3}+x^{2}+x$
            \item $g_{14}(x) = x^{3}+x^{2}+1$
            \item $g_{15}(x) = x^{3}+x^{2}+x+1$
        \end{itemize}
    \end{multicols}

    
    Su respectiva tabla de suma es:
    \begin{table}[h]
    \begin{center}
    \begin{tabular}{|c|c|c|c|c|c|c|c|c|c|c|c|c|c|c|c|c|}
    \hline
    $+$ & $g_{0}$ & $g_{1}$ & $g_{2}$ & $g_{3}$ & $g_{4}$ & $g_{5}$ & $g_{6}$ &
    $g_{7}$ & $g_{8}$ & $g_{9}$ & $g_{10}$ & $g_{11}$ & $g_{12}$ & $g_{13}$ &
    $g_{14}$ & $g_{15}$ \\ \hline
    $g_{0}$ & $g_{0}$ & $g_{1}$ & $g_{2}$ & $g_{3}$ & $g_{4}$ & $g_{5}$ & 
    $g_{6}$ & $g_{7}$ & $g_{8}$ & $g_{9}$ & $g_{10}$ & $g_{11}$ & $g_{12}$ & 
    $g_{13}$ & $g_{14}$ & $g_{15}$  \\ \hline
    $g_{1}$ & $g_{1}$ & $g_{0}$ & $g_{3}$ & $g_{2}$ & $g_{5}$ & $g_{4}$ &  
    $g_{7}$ & $g_{6}$ & $g_{9}$ & $g_{8}$ & $g_{11}$ & $g_{10}$ & $g_{14}$ & 
    $g_{15}$ & $g_{12}$ & $g_{13}$\\ \hline
    $g_{2}$ & $g_{2}$ & $g_{3}$ & $g_{0}$ & $g_{1}$ & $g_{6}$ & $g_{7}$ &
    $g_{4}$ & $g_{5}$ & $g_{10}$ & $g_{11}$ & $g_{8}$ & $g_{9}$ & $g_{13}$ &
    $g_{12}$ & $g_{15}$ & $g_{14}$ \\ \hline
    $g_{3}$ & $g_{3}$ & $g_{2}$ & $g_{1}$ & $g_{0}$ & $g_{7}$ & $g_{6}$ &
    $g_{5}$ & $g_{4}$ & $g_{11}$ & $g_{10}$ & $g_{9}$ & $g_{8}$ & $g_{15}$ &
    $g_{14}$ & $g_{13}$ & $g_{12}$ \\ \hline
    $g_{4}$ & $g_{4}$ & $g_{5}$ & $g_{6}$ & $g_{7}$ & $g_{0}$ & $g_{1}$ &
    $g_{2}$ & $g_{3}$ & $g_{12}$ & $g_{14}$ & $g_{13}$ & $g_{15}$ & $g_{8}$ & 
    $g_{10}$ & $g_{9}$ & $g_{11}$ \\ \hline
    $g_{5}$ & $g_{5}$ & $g_{4}$ & $g_{7}$ & $g_{6}$ & $g_{1}$ & $g_{0}$ & 
    $g_{3}$ & $g_{2}$ & $g_{14}$ & $g_{12}$ & $g_{15}$ & $g_{13}$ & $g_{9}$ &
    $g_{11}$ & $g_{8}$ & $g_{10}$ \\ \hline
    $g_{6}$ & $g_{6}$ & $g_{7}$ & $g_{4}$ & $g_{5}$ & $g_{2}$ & $g_{3}$ & 
    $g_{0}$ & $g_{1}$ & $g_{13}$ & $g_{15}$ & $g_{12}$ & $g_{14}$ & $g_{10}$ & 
    $g_{8}$ & $g_{11}$ & $g_{9}$ \\ \hline
    $g_{7}$ & $g_{7}$ & $g_{6}$ & $g_{5}$ & $g_{4}$ & $g_{3}$ & $g_{2}$ &  
    $g_{1}$ & $g_{0}$ & $g_{15}$ & $g_{13}$ & $g_{14}$ & $g_{12}$ & $g_{11}$ &
    $g_{9}$ & $g_{10}$ & $g_{8}$ \\ \hline
    $g_{8}$ & $g_{8}$ & $g_{9}$ & $g_{10}$ & $g_{11}$ & $g_{12}$ & $g_{14}$ &  
    $g_{13}$ & $g_{15}$ & $g_{0}$ & $g_{1}$ & $g_{2}$ & $g_{3}$ & $g_{4}$ &
    $g_{6}$ & $g_{5}$ & $g_{7}$ \\ \hline
    $g_{9}$ & $g_{9}$ & $g_{8}$ & $g_{11}$ & $g_{10}$ & $g_{14}$ & $g_{12}$ &  
    $g_{15}$ & $g_{13}$ & $g_{1}$ & $g_{0}$ & $g_{3}$ & $g_{2}$ & $g_{5}$ & 
    $g_{7}$ & $g_{4}$ & $g_{6}$ \\ \hline
    $g_{10}$ & $g_{10}$ & $g_{11}$ & $g_{8}$ & $g_{9}$ & $g_{13}$ & $g_{15}$ &  
    $g_{12}$ & $g_{14}$ & $g_{2}$ & $g_{3}$ & $g_{0}$ & $g_{1}$ & $g_{6}$ & 
    $g_{4}$ & $g_{7}$ & $g_{5}$ \\ \hline
    $g_{11}$ & $g_{11}$ & $g_{10}$ & $g_{9}$ & $g_{8}$ & $g_{15}$ & $g_{13}$ &  
    $g_{14}$ & $g_{12}$ & $g_{3}$ & $g_{2}$ & $g_{1}$ & $g_{0}$ & $g_{7}$ &
    $g_{5}$ & $g_{6}$ & $g_{4}$ \\ \hline
    $g_{12}$ & $g_{12}$ & $g_{14}$ & $g_{13}$ & $g_{15}$ & $g_{8}$ & $g_{9}$ &  
    $g_{10}$ & $g_{11}$ & $g_{4}$ & $g_{5}$ & $g_{6}$ & $g_{7}$ & $g_{0}$ & 
    $g_{2}$ & $g_{1}$ & $g_{3}$ \\ \hline
    $g_{13}$ & $g_{13}$ & $g_{15}$ & $g_{12}$ & $g_{14}$ & $g_{10}$ & $g_{11}$&
    $g_{8}$ & $g_{9}$ & $g_{6}$ & $g_{7}$ & $g_{4}$ & $g_{5}$ & $g_{2}$ &
    $g_{0}$ & $g_{3}$ & $g_{1}$ \\ \hline
    $g_{14}$ & $g_{14}$ & $g_{12}$ & $g_{15}$ & $g_{13}$ & $g_{9}$ & $g_{8}$ &  
    $g_{11}$ & $g_{10}$ & $g_{5}$ & $g_{4}$ & $g_{7}$ & $g_{6}$ & $g_{1}$ &  
    $g_{3}$ & $g_{0}$ & $g_{2}$ \\ \hline
    $g_{15}$ & $g_{15}$ & $g_{13}$ & $g_{14}$ & $g_{12}$ & $g_{11}$ & $g_{10}$&
    $g_{9}$ & $g_{8}$ & $g_{7}$ & $g_{6}$ & $g_{5}$ & $g_{4}$ & $g_{3}$ & 
    $g_{1}$ & $g_{2}$ & $g_{0}$\\ \hline
    \end{tabular}
    \end{center}
    \end{table}
    
    \newpage
    mientras que su tabla de multiplicación es:
    \begin{table}[h]
    \begin{center}
    \begin{tabular}{|c|c|c|c|c|c|c|c|c|c|c|c|c|c|c|c|c|}
    \hline
    $\cdot$ & $g_{0}$ & $g_{1}$ & $g_{2}$ & $g_{3}$ & $g_{4}$ &
    $g_{5}$ & $g_{6}$ & $g_{7}$ & $g_{8}$ & $g_{9}$ & 
    $g_{10}$ & $g_{11}$ & $g_{12}$ & $g_{13}$ & $g_{14}$ & 
    $g_{15}$ \\ \hline
    $g_{0}$ & $g_{0}$ & $g_{0}$ & $g_{0}$ & $g_{0}$ & 
    $g_{0}$ & $g_{0}$ & $g_{0}$ & $g_{0}$ & $g_{0}$ & $g_{0}$  
    & $g_{0}$ & $g_{0}$ & $g_{0}$ & $g_{0}$ & $g_{0}$ & 
    $g_{0}$ \\ \hline
    $g_{1}$ & $g_{0}$ & $g_{1}$ & $g_{2}$ & $g_{3}$ & 
    $g_{4}$ & $g_{5}$ & $g_{6}$ & $g_{7}$ & $g_{8}$ & 
    $g_{9}$ & $g_{10}$ & $g_{11}$ & $g_{12}$ & $g_{13}$ & 
    $g_{14}$ & $g_{15}$ \\ \hline
    $g_{2}$ & $g_{0}$ & $g_{2}$ & $g_{4}$ & $g_{6}$ & $g_{8}$ & $g_{10}$ & 
    $g_{12}$ & $g_{13}$ & $g_{3}$ & $g_{1}$ & $g_{7}$ & $g_{5}$ & $g_{11}$ & 
    $g_{15}$& $g_{9}$ & $g_{14}$ \\ \hline
    $g_{3}$ & $g_{0}$ & $g_{3}$ & $g_{6}$ & $g_{5}$ & $g_{12}$ & $g_{15}$ & 
    $g_{10}$ & $g_{9}$ & $g_{11}$ & $g_{8}$ & $g_{14}$ & $g_{13}$ & $g_{7}$ & 
    $g_{1}$ & $g_{4}$ & $g_{2}$ \\ \hline
    $g_{4}$ & $g_{0}$ & $g_{4}$ & $g_{8}$ & $g_{12}$ & $g_{3}$ & $g_{7}$ & 
    $g_{11}$ & $g_{15}$ & $g_{6}$ & $g_{2}$ & $g_{13}$ & $g_{10}$ & $g_{5}$ & 
    $g_{14}$ & $g_{1}$ & $g_{9}$ \\ \hline
    $g_{5}$ & $g_{0}$ & $g_{5}$ & $g_{10}$ & $g_{15}$ & $g_{7}$ & $g_{2}$ & 
    $g_{14}$ & $g_{8}$ & $g_{13}$ & $g_{11}$ & $g_{4}$ & $g_{1}$ & $g_{9}$ & 
    $g_{3}$ & $g_{12}$ & $g_{6}$ \\\hline
    $g_{6}$ & $g_{0}$ & $g_{6}$ & $g_{12}$ & $g_{10}$ & $g_{11}$ & $g_{14}$ & 
    $g_{7}$ & $g_{1}$ & $g_{5}$ & $g_{3}$ & $g_{9}$ & $g_{15}$ & $g_{13}$ & 
    $g_{2}$ & $g_{8}$ & $g_{4}$ \\\hline
    $g_{7}$ & $g_{0}$ & $g_{7}$ & $g_{13}$ & $g_{9}$ & $g_{15}$ & $g_{8}$ & 
    $g_{1}$ & $g_{6}$ & $g_{14}$ & $g_{10}$ & $g_{3}$ & $g_{4}$ & $g_{2}$ & 
    $g_{12}$ & $g_{5}$ & $g_{11}$ \\\hline
    $g_{8}$ & $g_{0}$ & $g_{8}$ & $g_{3}$ & $g_{11}$ & $g_{6}$ & $g_{13}$ & 
    $g_{5}$ & $g_{14}$ & $g_{12}$ & $g_{4}$ & $g_{15}$ & $g_{7}$ & $g_{10}$ & 
    $g_{9}$ & $g_{2}$ & $g_{1}$ \\ \hline
    $g_{9}$ & $g_{0}$ & $g_{9}$ & $g_{1}$ & $g_{8}$ & $g_{2}$ & $g_{11}$ & 
    $g_{3}$ & $g_{10}$ & $g_{4}$ & $g_{14}$ & $g_{5}$ & $g_{12}$ & $g_{6}$ & 
    $g_{7}$ & $g_{15}$ & $g_{13}$ \\ \hline
    $g_{10}$ & $g_{0}$ & $g_{10}$ & $g_{7}$ & $g_{14}$ & $g_{13}$ & $g_{4}$ & 
    $g_{9}$ & $g_{3}$ & $g_{15}$ & $g_{5}$ & $g_{8}$ & $g_{2}$ & $g_{1}$ & 
    $g_{6}$ & $g_{11}$ & $g_{12}$ \\ \hline
    $g_{11}$ & $g_{0}$ & $g_{11}$ & $g_{5}$ & $g_{13}$ & $g_{10}$ & $g_{1}$ & 
    $g_{15}$ & $g_{4}$ & $g_{7}$ & $g_{12}$ & $g_{2}$ & $g_{9}$ & $g_{14}$ &
    $g_{8}$ & $g_{6}$ & $g_{3}$ \\ \hline
    $g_{12}$ & $g_{0}$ & $g_{12}$ & $g_{11}$ & $g_{7}$ & $g_{5}$ & $g_{9}$ & 
    $g_{13}$ & $g_{2}$ & $g_{10}$ & $g_{6}$ & $g_{1}$ & $g_{14}$ & $g_{15}$ & 
    $g_{4}$ & $g_{3}$ & $g_{8}$ \\ \hline
    $g_{13}$ & $g_{0}$ & $g_{13}$ & $g_{15}$ & $g_{1}$ & $g_{14}$ & $g_{3}$ & 
    $g_{2}$ & $g_{12}$ & $g_{9}$ & $g_{7}$ & $g_{6}$ & $g_{8}$ & $g_{4}$ & 
    $g_{11}$ & $g_{10}$ & $g_{5}$ \\ \hline
    $g_{14}$ & $g_{0}$ & $g_{14}$ & $g_{9}$ & $g_{4}$ & $g_{1}$ & $g_{12}$ & 
    $g_{8}$ & $g_{5}$ & $g_{2}$ & $g_{15}$ & $g_{11}$ & $g_{6}$ & $g_{3}$ & 
    $g_{10}$ & $g_{13}$ & $g_{7}$  \\ \hline
    $g_{15}$ & $g_{0}$ & $g_{15}$ & $g_{14}$ & $g_{2}$ & $g_{9}$ & $g_{6}$ & 
    $g_{4}$ & $g_{11}$ & $g_{1}$ & $g_{13}$ & $g_{12}$ & $g_{3}$ & $g_{8}$ & 
    $g_{5}$ & $g_{7}$ & $g_{10}$  \\ \hline
    \end{tabular}
    \end{center}
    \end{table}
    
    % Ejercicio 2.
    \item Construye una matriz generadora para el código $RS(4,11)$. 
    
    \textsc{Solución:} Una matriz generadora para $RS(4, 11)$ es 
    \begin{equation*}
        G = 
        \begin{pmatrix}
        1 & 0 & 0 & 0 & 0 & 0 & 1 & 1 & 1 & 1 & 1\\
        0 & 1 & 0 & 0 & 1 & 0 & 0 & 1 & 1 & 0 & 1\\
        0 & 0 & 1 & 0 & 0 & 1 & 1 & 1 & 0 & 0 & 1\\
        0 & 0 & 0 & 1 & 1 & 1 & 0 & 0 & 0 & 1 & 1
        \end{pmatrix}
    \end{equation*}
    
    \newpage
    % Ejercicio 3.
    \item Supón que recibes la palabra $y = (10, 1, 2, 2, 2, 10, 7, 2, 9, 3, 7)$
    $\in \mathbb{F}^{11}_{11}$. Decodifica la palabra usando el algoritmo de
    Gao, sabiendo que la palabra es del código $RS(4,11)$.

    % Ejercicio 4.
    \item Construye una base para $\mathcal{L}_{k}$ de tal manera que la matriz
    generadora del código $RS(k, q)$ sea de la forma 
    \begin{equation}
    \begin{bmatrix}
    I_{k} & P\\
    \end{bmatrix}
    \end{equation}
    
    donde $I_{k}$ es la matriz identidad $k \times k$ y $P$ es una matriz 
    $k \times (q-k)$. 
    
    \textsc{Solución:} Sabemos que $\mathcal{L}_{k} = \{f(x) \in 
    \mathbb{F}_{q}[x] \; : \; deg(f) < k\}$, es decir, $\mathcal{L}_{k}$ son 
    todos los polinomios de grado menor que $k$ con coeficientes en 
    $\mathbb{F}_{q} = \{0, 1, \alpha, \alpha^{2}, ..., \alpha^{q-2}\}$, donde 
    éste es un campo finito de $q$ elementos. 

    Sea $\beta = \{1, \alpha x, \alpha_{2} x^{2}, ..., \alpha_{k-1} x^{k-1}\}$,
    con $\alpha_{i} \in \mathbb{F}_{q} - \{0\}$ e $i \in \{1, 2, ..., q - 2\}$.
    Entonces tenemos que $\beta$ es una base para $\mathcal{L}_{k}$ con una 
    matriz generadora 
    \begin{equation*}
        G = 
        \begin{bmatrix}
        I_{k} & P\\
        \end{bmatrix}
    \end{equation*}

    Recordemos que un mensaje $m = (m_1, m_2, ..., m_k)$ se codifica haciendo 
    $m \cdot G$. Entonces 
    \begin{equation*}
        G = 
        \begin{bmatrix}
        I_{k} & P\\
        \end{bmatrix}
        =  
        \begin{bmatrix}
            1 & 0 & \cdots & 0 & & 1 & 0 & 0 & \cdots & 0 \\ 
            0 & 1 & \cdots & 0 & & 0 & \alpha x & 0 & \cdots & 0 \\
            \vdots & \cdots & \ddots & \vdots & & \vdots & \vdots & 
            \alpha_{2} x^{2} & \vdots & 0 \\ 
            0 & 0 & \cdots & 1 & & 0 & 0 & 0 & \cdots & \alpha_{k-1} x^{k-1}  
        \end{bmatrix}
    \end{equation*}

    y $m \cdot G = (m_1, m_2, ..., m_k, m_1, m_2 \alpha x, m_3 \alpha_2 x^2, ..., 
    m_k \alpha_{k-1} x^{k-1})$. 

    Así, $G$ tiene dimensión $k$ y cada mensaje codificado tiene 
    longitud $q$. Por lo tanto, $G$ genera a $RS(k, q)$. 
    
    % Ejercicio 5.
    \item Demuestra que el número de subespacios vectoriales de 
    $\mathbb{F}^{n}_{q}$ de dimensión $i$ es: 
    \begin{equation}
        \mathcal{G}(n, i) = \frac{(q^{n}-1)(q^{n}-q)\cdots(q^{n}-q^{i-1})}
        {(q^{i}-1)(q^{i}-q)\cdots(q^{i}-q^{i-1})}
    \end{equation}
    
    para $i = 1,.., n$.
    
    \begin{proof}
        Sabemos que un subespacio de dimensión $k$ se especifica dando $k$
        vectores linealmente independientes $\{v_{1}, v_{2}, ..., v_{k}\}
        \in V = \mathbb{F}^{n}_{q}$. El vector $v_{1}$ puede ser elegido como 
        cualquier vector distinto de cero en $V$, por lo que hay $q^{n}-1$ 
        opciones para $v_{1}$. Dado $v_{1}$, $v_{2}$ se puede elegir como 
        cualquier vector que no se encuentre en el subespacio generado por 
        $v_{1}$. Como este subespacio tiene $q$ elementos, entonces hay 
        $q^{n} - q$ opciones para $v_{2}$. Siguiendo de esta manera, tenemos 
        que dados $v_{1}, v_{2},..., v_{i}$ con $i < k$, entonces hay 
        $q^{n}-q^{i}$ opciones para $v_{i+1}$. Así, el número de conjuntos de 
        $k$ vectores linealmente independientes en $V$ es
        \begin{equation}
            (q^{n}-1)(q^{n}-q)\cdots(q^{n}-q^{k-1})
        \end{equation}
        
        Como hay muchos conjuntos $k$ linealmente independientes que
        generan el mismo subespacio, entonces debemos dividir la expresión
        anterior entre el número de $k$ conjuntos que generan el mismo
        subespacio (i.e. el número de bases para un subespacio de dimensión 
        $k$).
        
        Así, aplicando la expresión $(4)$ al caso especial en que $n = k$, 
        tenemos que cada subespacio de dimensión $k$ de $V$ tiene
        \begin{equation}
            (q^{k}-1)(q^{k}-q)\cdots(q^{k}-q^{k-1})
        \end{equation}
        
        \newpage
        bases. Por lo tanto, el número de subespacios de dimensión $k$ de $V$ 
        es 
        \begin{center}
            $\mathcal{G}(n, k) = \frac{(q^{n}-1)(q^{n}-q)\cdots(q^{n}-q^{k-1})}
                                    {(q^{k}-1)(q^{k}-q)\cdots(q^{k}-q^ {k-1})}$
        \end{center}
    \end{proof}

    % Ejercicio 6.
    \item Demuestra que $RS(k, q)^{\top}_{q} = RS(q-k, q)$.
    \begin{proof}
        Notemos que tanto $RS(k,q)$ como $RS(q-k,q)$ tienen dimensiones 
        complementarias, por lo que es suficiente mostrar que son ortogonales.
        Tenemos que el código $RS(k,q)$ es generado por el polinomio $X^{i}$,
        con $i < k$; y el código $RS(q-k, q)$ es generado por el 
        polinomio $X^{j}$, con $j < q-k$. El producto punto de los 
        correspondientes códigos es $\sum_{v} v^{i+j}$ (teniéndo en cuenta que
        $i + j \leq q -2$). Sabemos que existen algunos elementos  
        $c \not = 0 \in \mathbb{F}_{q}$ tales que $c^{i+j} \not = 1$, pues de 
        lo contrario, el polinomio $X^{i+j}-1$ de grado $i+j$ debería de tener
        $q-1 > i+j$ raíces, lo cual es imposible.
        Además, tenemos que $c^{i+j} \left(\sum_{v} v^{i+j} \right) = 
        \sum_{v} v^{i+j}$, ya que $cv$ se \textit{desplaza} a través de todos
        los elementos distintos de cero que pertenecen a $\mathbb{F}_{q}$ 
        cuando $v$ lo hace. De aquí se sigue que $(c^{i+j}-1) \left(\sum_{v} 
        v^{i+j} \right) = 0$, y como $c^{i+j} \not = 1$, entonces podemos
        concluir que $\sum_{v} v^{i+j} = 0$ (que es lo que queríamos mostrar).
        Por lo tanto, $RS(k, q)^{\top}_{q} = RS(q-k, q)$.
         
    \end{proof}
        
    % Ejercicio 7.
    \item Demuestra que si $C$ es un código $MDS$, entonces $C^{\top}$ también
    es $MDS$. 
    \begin{proof}
        Supongamos que $C$ es un código $MDS$ con una matriz generadora $G$ 
        con columnas $c_{i}$, donde $i \in \{1,2, ..., n\}$. Entonces $G$ es
        una matriz de $k \times n$, y por hipótesis tenemos que cada 
        combinación lineal de los renglones tiene un peso de Hamming de al
        menos $n-k+1$. Por una proposición vista en clase, sabemos que 
        $C^{\top}$ tiene una matriz de verificación de paridad $G^{T}$.
        Como $C = [n,k,n-k+1]$, entonces debemos mostrar que la distancia
        mínima de $C^{\top}$ es igual a $n-(n-k)+1= k+1$, es decir, 
        $C^{\top} = [n,n-k+1,k+1]$ (esto significa que cada subconjunto de $k$ 
        columnas de la matriz generadora $G$ es linealmente independiente).
        
        Procedemos por contradicción. Supongamos que algunas $k$ columnas de 
        $G$ son linealmente dependientes. Sea $H$ la submatriz de 
        $k \times k$ formada por estas columnas. Como las columnas son 
        linealmente dependientes, entonces el rango de $H$ es menor que 
        $k$ debido a que los renglones de $H$ tienen alguna dependencia lineal. 
        Por lo tanto, existe una combinación lineal
        de los renglones de $H$ que suma $0$, por lo que podemos usar esta 
        misma combinación lineal en los renglones de $G$ cuya suma tiene al
        menos $k$ ceros, lo cual implicaría que tiene un peso de Hamming 
        $\leq n -k$. Pero como cualquier combinación lineal de los renglones
        de $G$ en un código $MDS$ debe tener un peso de Hamming al menos 
        $n-k+1$, entonces tenemos una contradicción.
        
        Por lo tanto, $C^{\top}$ es un código $MDS$.
        
    \end{proof}

    % Ejercicio 8.
    \item Resuelve los siguientes ejercicios
    \begin{enumerate}
        \item Encuentra la matriz generadora $G$ del código Simplex $S(3,2)$.
        
        \textsc{Solución:} Sabemos que el código $S(3,2)$ tiene 
        \begin{align*}
            \frac{q^{k}-1}{q-1} 
            &= \frac{2^{3}-1}{2-1} \\
            &= \frac{8-1}{1} \\
            &= 7
        \end{align*}
        
        subespacios de dimensión $1$. Por lo tanto, 

        \begin{equation*}
        G = 
        \begin{pmatrix}
        1 & 0 & 0 & 1 & 1 & 0 & 1\\
        0 & 1 & 0 & 1 & 0 & 1 & 1\\
        0 & 0 & 1 & 0 & 1 & 1 & 1
        \end{pmatrix}
        \end{equation*}
        
        \item Supongamos que un mensaje es enviado bajo el código $H(3,2)$.
        Verifica si el mensaje $r = 1010001$ es correcto. 
        
        \textsc{Solución:} Sabemos que $H(3,2) = [7, 4, 3]_{2}$ y que una
        matriz de verificación para $H(3,2)$ es cualquier matriz generadora
        para $S(3,2)$. Así, la matriz obtenida en el inciso anterior es una
        matriz de verificación para nuestro código $H(3,2)$.
        
        El mensaje $r$ se puede ver como un vector 
        \begin{equation*}
            x = (1, 0, 1, 0, 0, 0, 1) \in \mathbb{F}^{7}_{2}
        \end{equation*}

        Ahora, calculamos el síndrome de $x$.
        \begin{align*}
            S(x) 
            &= G x^{t} \\
            &= \begin{pmatrix}
               1 & 0 & 0 & 1 & 1 & 0 & 1\\
               0 & 1 & 0 & 1 & 0 & 1 & 1\\
               0 & 0 & 1 & 0 & 1 & 1 & 1
               \end{pmatrix}
               \begin{pmatrix}
               1 \\
               0 \\
               1 \\
               0 \\ 
               0 \\
               0 \\
               1
               \end{pmatrix} \\
            &= \begin{pmatrix}
               0 \\
               1 \\
               0 
               \end{pmatrix} \\
        \end{align*}
        
        Como $S(x) = (0, 1, 0)^{t} \neq 0$ entonces podemos concluir que hubo
        errores de transmisión. Notemos que $(0, 1, 0)^{t}$ corresponde a la 
        segunda columna de $G$, por lo que sabemos que la segunda coordenada es
        incorrecta. Entonces el vector error es
        \begin{center}
            $e = (0, 1, 0, 0, 0, 0, 0)$
        \end{center}
        
        Por lo tanto, la palabra envíada fue
        \begin{align*}
            z &= x - e \\
            &= (1, 0, 1, 0, 0, 0, 1) - (0, 1, 0, 0, 0, 0, 0) \\
            &= (1, 1, 1, 0, 0, 0, 1)
        \end{align*}
    \end{enumerate}
\end{enumerate}
\end{document}
